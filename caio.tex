\documentclass{article}
\usepackage[brazil]{babel}
\usepackage[utf8]{inputenc}

\usepackage{enumitem} %for doing \begin{enumerate}[(i)] 

\usepackage{amsmath}
\usepackage{amssymb}
\usepackage{amsthm}

\newtheorem{problema}{Problema}
\newtheorem{exercicio}{Exercício}

\newcommand{\NN}{\mathbb{N}}
\newcommand{\rad}{\operatorname{rad}}

\begin{document}
	
	\title{Exercícios Atiyah-Macdonald}
	
	
	
	\maketitle
	
	Soluções por Caio Antony G de M Andrade
	
	\exercicio Let $x$ be a nilpotent element of a ring $A$. Show that $1+x$ is a unit of $A$. Deduce that the sum of a nilpotent element and a unit is a unit.
	
	\begin{proof}
		Seja $x\in A$ um nilpotente qualquer, e $y = -x$, que também é nilpotente. Sendo $n\in\NN$ o menor natural tal que $y^n = 0$, temos
		\[
			(1 - y)(1 + y + \dotsb + y^{n-1}) = 1 - y^n = 1\,,
		\]
		de onde $1 - y = 1 + x$ é invertível.\\ 
		Se $\lambda\in A$ é também invertível, então $\lambda^{-1}x$ é nilpotente, de onde $1 + \lambda^{-1}x$ é inversível, e portanto $\lambda + x = \lambda(1 + \lambda^{-1}x)$ é inversível.
	\end{proof}
	
	
	\exercicio Let $A$ be a ring and let $A[x]$ be the ring of polynomials in an indeterminate $x$, with coefficients in $A$. Let $f = a_0 + a_1x + \dotsb + a_nx^n\in A[x]$. Prove that
	\begin{enumerate}[label = \roman*)]
	\item \label{1} $f$ is a unit in $A[x] \Leftrightarrow a_{0}$ is a unit in $A$ and $a_{1}, \ldots, a_{n}$ are nilpotent. [If $b_{0}+b_{1} x+\cdots+b_{m} x^{m}$ is the inverse of $f$, prove by induction on $r$ that $a_{n}^{r+1} b_{m-r}=0 .$ Hence show that $a_{n}$ is nilpotent, and then use Ex. 1.]
	\item \label{2} $f$ is nilpotent $\Leftrightarrow a_{0}, a_{1}, \ldots, a_{n}$ are nilpotent.
	\item \label{3} $f$ is a zero-divisor $\Leftrightarrow$ there exists $a \neq 0$ in $A$ such that $a f=0 .$ [Choose a polynomial $g=b_{0}+b_{1} x+\cdots+b_{m} x^{m}$ of least degree $m$ such that $f g=0$. Then $a_{n} b_{m}=0$, hence $a_{n} g=0$ (because $a_{n} g$ annihilates $f$ and has degree $<m$). Now show by induction that $a_{n-r} g=0$ ($\, 0 \leqslant r \leqslant n$).]
	\item \label{4} $f$ is said to be primitive if $\left(a_{0}, a_{1}, \ldots, a_{n}\right)=(1) .$ Prove that if $f, g \in A[x]$, then $f g$ is primitive $\Leftrightarrow f$ and $g$ are primitive.
	\end{enumerate}
	
	\begin{proof}
		too long
		
		\ref{1} ($\Rightarrow$) Suponha que $g(x) = b_0 + b_1x + \dotsb b_mx^m$ é o inverso de $f$. Temos que 
		\[
			1 = gf = \sum_{i=1}^{m+n}c_ix^i\,, \quad\text{com}\; c_i = \sum_{j=0}^{i}a_jb_{i-j}\,,
		\]
		e note que $c_0 = 1$ e $c_1 = 0$ para $i>0$. Assim vemos que $a_0$ e $b_0$ são inversíveis. Vemos também em particular, de $c_{m+n} = 0$, que $a_nb_m = 0$. Ainda, multiplicando $c_{m+n-1} = a_nb_{m-1} + a_{n-1}b_m$ por $a_n$ e usando que $a_nb_m = 0$, vemos que $a_n^2b_{m-1}=0$. Prosseguindo indutivamente, obtemos 
	\end{proof}
	
	\exercicio Generalize the results of Exercise 2 to a polynomial ring $A\left[x_{1}, \ldots, x_{r}\right]$ in several indeterminates.
	\exercicio In the ring $A[x]$, the Jacobson radical is equal to the nilradical.
	\exercicio Let $A$ be a ring and let $A[[x]]$ be the ring of formal power series $f=\sum_{n=0}^{\infty} a_{n} x^{n}$ with coefficients in $A$. Show that
	i) $f$ is a unit in $A[[x]] \Leftrightarrow a_{0}$ is a unit in $A$.
	ii) If $f$ is nilpotent, then $a_{n}$ is nilpotent for all $n \geqslant 0 .$ Is the converse true? (See Chapter 7 , Exercise 2.)
	iii) $f$ belongs to the Jacobson radical of $A[[x]] \Leftrightarrow a_{0}$ belongs to the Jacobson radical of $A .$
	iv) The contraction of a maximal ideal $\mathrm{m}$ of $A[[x]]$ is a maximal ideal of $A$, and $\mathrm{m}$ is generated by $\mathrm{m}^{\varepsilon}$ and $x .$
	v) Every prime ideal of $A$ is the contraction of a prime ideal of $A[[x]]$
	\exercicio A ring $A$ is such that every ideal not contained in the nilradical contains a nonzero idempotent (that is, an element e such that $e^{2}=e \neq 0$ ). Prove that the nilradical and Jacobson radical of $A$ are equal.
	
	\begin{proof}
		Suponha que $\rad(A)\not\subseteq N(A)$, e tome $a\in \rad(A)\backslash N(A)$. Como $(a)\not\subseteq N(A)$, existe $b\in A$ tal que $0\neq ab = abab$. Como $a\in \rad(A)$, tem-se que $1 - ab\in U(A)$. Mas
		\[
			(1-ab)ab = 0\,,
		\]
		o que é uma contradição, pois um elemento não pode ser divisor de zero e inversível simultaneamente. Assim, $\rad(A)\subseteq N(A)$, o que mostra o resultado.
	\end{proof}

	
	\exercicio Let $A$ be a ring in which every element $x$ satisfies $x^{n}=x$ for some $n>1$ (depending on $x$). Show that every prime ideal in $A$ is maximal.
	\exercicio Let $A$ be a ring $\neq0$. Show that the set of prime ideals of A has minimal elements with respect to inclusion. 
	\exercicio Let $\mathfrak{a}$ be an ideal $\neq(1)$ in a ring $A$. Show that $\mathfrak{a} = r(\mathfrak{a})\iff \mathfrak{a}$ is an intersection of prime ideals.
	\exercicio Let $A$ be a ring, $\Re$ its nilradical. Show that the following are equivalent:
	i) $A$ has exactly one prime ideal;
	ii) every element of $A$ is either a unit or nilpotent;
	iii) $A / \Re$ is a field.
	\exercicio A ring $A$ is Boolean if $x^{2}=x$ for all $x \in A$. In a Boolean ring $A$, show that
	i) $2 x=0$ for all $x \in A$;
	ii) every prime ideal $p$ is maximal, and $A / \mathfrak{p}$ is a field with two elements;
	iii) every finitely generated ideal in $A$ is principal.
	\exercicio A local ring contains no idempotent $\neq 0,1$.
	
	\begin{proof}
		Seja $e\in A$ um idempotente. Como $A =  \rad(A)\cup U(A)$ é local, $e\in\rad(A)$ ou $e\in U(A)$. Suponha $e\neq0,1$. Temos de $e(e-1) = 0$ que $e$ é divisor de zero, e assim, $e\in\rad(A)$. Temos que $1-e\in U(A)$. %Seja $y\in A$ o inverso de $1-e$. Então
%		\[
%			(1-e)y = 1 \Rightarrow 0 = e(1-e)y = e\,,
%		\]
		Mas $1-e$ também é divisor de zero, absurdo.
	\end{proof}
	
\end{document}